\chapter{Einleitung}
\section{Motivation}
Videospiele haben eine lange Entwicklung hinter sich. Nicht nur haben sich Gameplay und Storytelling in den letzten Jahrzehnten weiterentwickelt, sondern vor allem die Grafik hat sehr große Fortschritte gemacht. Ältere Spiele wie \textit{Pong} (Atari, 1972) bestanden nur aus weißen Strichen und Punkten vor einem schwarzen Hintergrund, heutige Titel wie \textit{God of War}  (Santa Monica Studio, 2018) und \textit{Far Cry New Dawn} (Ubisoft Montreal, 2019) sind der Realität näher als je zuvor. Eine Methode, die genutzt wird, um die immer größeren und imposanteren Welten zu erschaffen ist Modularität \parencite{Burgess}.
\par
Modularität wird in der Videospielentwicklung in vielen Bereichen auf verschiedene Arten genutzt:
\begin{itemize}
\item Für das Erzeugen von Musik, die immer zum aktuellen Geschehen passt \parencite[S.\,12]{music},
\item in der Programmierung, wo einmal programmierte modulare Komponenten auf verschiedene Arten genutzt werden, um diverse Aufgaben zu übernehmen \parencite{modcode},
\item für die Generierung von Items, um eine große Auswahl mit individuellen Eigenschaften zu erschaffen \parencite{Borderlands},
\item für die Entwicklung von Modellen, bei der vorgefertigte Teile immer wieder eingesetzt werden, um den Arbeitsprozess zu beschleunigen \parencite{ForHonor},
\item und das Erzeugen von ganzen Leveln oder Levelelementen, die direkt in der Game Engine zusammengefügt werden können \parencite{Burgess}.
\end{itemize}
\par
Letzteres ist für \textit{Raw Vengeance Studios}, das Unternehmen in dessen Kontext diese Thesis ausgearbeitet wird, besonders interessant. Das Start-up entwickelt einen Third Person Cartoon Shooter namens \textit{Renegade Line}. Ich bin Aktuell der einzige 3D-Artist der an der Entwicklung beteiligt ist und kann nicht genügend Modelle produzieren, um beide Leveldesigner sinnvoll zu beschäftigen. Ein Aspekt der Modularität, welcher von großem Vorteil für das Unternehmen wäre, ist, dass es die Möglichkeit bieten soll, mit wenigen 3D-Artists viel Inhalt zu generieren \parencite{Burgess}. Dabei werden verhältnismäßig wenige von Artists erstellte Modelle von Leveldesignern genutzt, um viel und abwechslungsreichen Content zu erstellen \parencite{Burgess}.
\newpage
\section{Zielsetzung}
In dieser Arbeit soll aufgezeigt werden, welche Methoden für die Erstellung modularer Assets genutzt werden und welche Vorteile durch den Einsatz von modularem Design in Videospielen erzielt werden können. Dabei wird ein Set aus Modellen erstellt, mit dessen Hilfe eine Vielzahl an unterschiedlichen Gebäuden ohne Innenräume generiert werden kann.
\par
Zu diesem Zweck werden zunächst Anwendungen von modularem Design in der Videospielentwicklung vorgestellt und es wird diskutiert, wie diese sich im Verlauf der Zeit entwickelt haben. Des Weiteren werden, anhand von LEGO und Gebäudebau, Beispiele aus der Realität dargestellt, in denen Modularität Vielfältigkeit und Effizienz ermöglicht. Für ein besseres Verständnis der modularen Methoden werden im Anschluss Grundlagen der Asset-Erstellung erörtert. Darauf aufbauend werden die Methoden, mit denen modulares Design ermöglicht wird, aufgelistet und erklärt. Es wird außerdem darauf eingegangen, wie modulare Modelle bewertet werden können.
\par
Nachdem alle theoretischen Aspekte abgehandelt wurden, werden die erarbeiteten Methoden in einem Projekt angewendet. In dem Projekt werden Modelle mit \textit{Blender} erstellt und in \textit{Unreal Engine} implementiert. Mithilfe der zuvor erarbeiteten Kriterien werden im Anschluss die erstellten Modelle bewertet. Auch die genutzten Methoden werden auf ihre Nützlichkeit hin ausgewertet.
\par
Abschließend wird ein Fazit aus den erlangten Informationen gebildet und ein Ausblick gewährt, welche Aspekte noch behandelt werden könnten und wie die Forschung diesbezüglich fortgesetzt wird.