\chapter{Fazit und Ausblick}
\section{Fazit}
Durch das Aufzeigen des Einflusses von Modularität auf die Entwicklung von Videospielen und die Realität konnte ein erster Eindruck von dessen Nutzen gewonnen werden. Die Erarbeitung des normalen Ablaufs der Asseterstellung und wie dieser durch spezielle Methoden auf die Anwendung von modularem Design angepasst wird haben gezeigt welche Tiefe dieses Konzept besitzt. Mithilfe der Anwendung der zuvor erarbeitetet Methoden und Konzepte konnten diese bewertet werden und eine Einschätzung von modularem Design erstellt werden.
\par
Eine ausführliche Bewertung der genutzten Methoden wurde schon im vorherigen Kapitel durchgeführt. Zusammenfassend lässt sich sagen, dass die wichtigsten Methoden, um modulares Design zu realisieren, die folgenden sind:
\begin{itemize}
\item Raster,
\item Pivot Point,
\item Planung,
\item Tests und
\item kachelbare Texturen.
\end{itemize}
Alle weiteren Methoden verbessern den Prozess deutlich, ohne die zuvor genannten ist das Umsetzen eines modularen Kits nicht möglich. Für andere Ansätze von Modularität kann diese Bewertung abweichen.
\par
Durch die vielen schon vorhandenen Methoden, welche trotz ihres Alters alle noch relevant sind, ist es leicht einen Einstieg in modulares Design zu finden. Ohne viel Zeit in die gründliche Planung zu investieren, lässt sich das Konzept jedoch nur schwer umsetzen.
\par
Auf Grund dieser großen Zeitinvestition vorab, muss für jedes Projekt geprüft werden, bis zu welchem Grad es sinnvoll ist Modularität zu nutzen. Lassen sich Assets nicht wieder verwerten oder wird immer nur eine geringe Anzahl ähnlicher Objekte genutzt sollte von einem modularen Kit abgesehen werden. Dennoch kann für die Erstellung der Assets, wie von L. Durand (siehe Abschnitt \ref{Geschichte von Modularität }) beschrieben, Modularität genutzt werden, um diesen Prozess zu beschleunigen. Dies wurde im praktischen Abschnitt dieser Arbeit nicht genau getestet. Da für die Erstellung der Varianten von Modulen aber ein ähnliches Vorgehen genutzt wurde, kann dies zu einem gewissen Umfang verifiziert werden.
\par
Das modulares Design einen hohen Nutzen hat, kann nur bestätigt werden. Mit Hilfe des erstellten Kits lassen sich beliebig viele Kombinationen an Gebäuden erstellen und durch die mögliche Erweiterung durch weitere Varianten der Module wird ein abwechslungsreiches Erscheinungsbild gewährt. Für alle weiteren Level, die dem gleichen Setting folgen, können Gebäude aus dem Kit erstellt werden. Dies bedeutet eine sehr große Zeitersparnis für die Zukunft, ohne dass immer die exakt gleichen Gebäude für die Level genutzt werden, wie es bisher der Fall war.
\par
Da bei diesem Projekt zum ersten Mal mit modularem Design gearbeitet wurde und wenig Praxiserfahrung vorhanden ist, können die Ergebnisse bei der Bearbeitung durch ein erfahreneres Team abweichen. Die, von vielen Autoren, angesprochene Performance-Verbesserung konnte nicht verifiziert werden. Dies Lag mit aller Wahrscheinlichkeit an der Nutzung einer falschen Modularitätsstufe oder einem anderen Fehler, der mit mehr Erfahrung vielleicht nicht passiert wäre. Die Effektivität der angewandten Methoden konnte dennoch gut eingeschätzt werden. 
\par
Tests bezüglich der visuellen Qualität des Ergebnisses waren auf Grund der zur Verfügung stehenden Mittel nur begrenzt möglich. Für eine bessere Evaluierung der visuellen  Qualität der Assets, hätte ein fertiges Level  mit den erstellten Assets von Testpersonen getestet werden müssen. Dies war aus Zeitgründen nicht möglich. 
\par
Die erarbeiteten Methoden und Ergebnisse können für ähnliche Projekte sicherlich hilfreich sein, auch wenn die Performance-Problematik nicht gelöst wurde.
\par
Die Zusammenfassung der Ergebnisse ist hier abgeschlossen, im folgenden Abschnitt wird darauf eingegangen, wie mit den Ergebnissen verfahren wird und ein Ausblick in die Zukunft präsentiert.
\section{Ausblick}
Grundlegend kann gesagt werden, dass die Möglichkeit modulares Design anzuwenden eine wichtige Erweiterung der Fähigkeiten eines 3D-Artists ist, um dem Trend der immer größeren und realistischeren Spielwelten gewachsen zu sein.
\par
Für die weitere Entwicklung von \textit{Renegade Line} wird versucht mit der Version 4.22 von Unreal Engine noch einmal den modularen Ansatz in Unreal Engine zu verfolgen. Mit dieser neuen Version wurde die Instanziierung von Objekten verbessert \parencite{unrealTalk}. Zuvor muss das Spiel allerdings auf diese Version portiert werden.
\par
Zudem ist geplant in der Zukunft Gebäude mit Innenräumen auszustatten, um ein komplexeres Spielerlebnis zu ermöglichen. Es bietet sich an für dieses Vorgehen Modularität zu nutzen. Mithilfe der erarbeiteten Erkenntnisse kann dies voraussichtlich leichter umgesetzt werden, auch wenn für dieses Anwendungsgebiet sicherlich Anpassungen vorgenommen werden müssen.
\par
Generell wäre es spannend die erlernten Methoden an anderen Projekten anzuwenden, die einem anderen Stil und unterschiedliche Stufen von Modularität folgen. Die Möglichkeiten, die dieses Thema bietet, sind, genau wie Modularität selbst, unbegrenzt.