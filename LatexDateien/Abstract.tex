\thispagestyle{empty}
\selectlanguage{english}
\section*{\centering\abstractname}
This thesis deals with the study and application of modular design in video game development. In particular, methods are developed which are used for the creation of a modular kit. The focus is on the generation of buildings without interiors. The basics of modularity and asset development are explained at the beginning of this thesis. In addition, methods are developed that enable the use of modular design. Furthermore criteria are developed which can be used to evaluate modular assets.
\par
By applying the methods, a set of assets could be created, with which theoretically infinitely many different buildings could be generated. The assets were created with Blender and implemented in Unreal Engine. Afterwards the applied methods could be evaluated by their practical use and benefit.
\par
The bachelor thesis addresses both beginners and advanced users of the application of modular design in asset development. In addition, the topic is interesting for smaller game studios to optimize asset development.
\selectlanguage{ngerman}
\section*{\centering\abstractname}
Diese Thesis beschäftigt sich mit der Untersuchung und Anwendung von modularem Design in der Videospielentwicklung. Im speziellen werden Methoden erarbeitet, die für die Erstellung eines modularen Kits eingesetzt werden. Der Fokus liegt dabei auf der Generierung von Gebäuden ohne Innenräume. Zu Beginn der Arbeit werden Grundlagen von Modularität und der Asset-Erstellung erläutert. Darüber hinaus werden Methoden erarbeitet, die den Einsatz von modularem Design ermöglichen. Es werden auch Kriterien erarbeitet, die modulare Assets bewerten können.
\par
Durch Anwendung der Methoden konnte ein Set von Assets angefertigt werden, mit dem theoretisch unendlich viele verschiedene Gebäude generiert werden können. Die Assets wurden mit Blender erstellt und in Unreal Engine implementiert. Die angewandten Methoden konnten nach dem praktischen Einsatz auf ihren Nutzen hin bewertet werden.
\par
Die Bachelorarbeit wendet sich sowohl an Einsteiger als auch an Fortgeschrittene der Anwendung von modularem Design in der Assetentwicklung. Zusätzlich ist das Thema für kleinere Game-Studios interessant, um die Asset-Entwicklung zu optimieren. 